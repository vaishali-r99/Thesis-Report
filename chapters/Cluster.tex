\chapter{Dimensionality reduction and clustering}
\label{chap:cluster}
In this chapter, we explore the application of unsupervised learning techniques, specifically dimensionality reduction and clustering, to analyze and interpret high-dimensional CFD simulation data of nozzle flows. Unsupervised learning, a class of machine learning methods that operates on unlabeled data, aims to discover hidden patterns or intrinsic structures within the data. Our objective is to distill high-dimensional simulation data into insightful, low-dimensional representations and identify distinct patterns through clustering, facilitating a deeper understanding of fluid behavior under various conditions. We use \gls{PCA} \cite{pearson1901pca}, \gls{t-SNE} \cite{vandermaaten2008tsne}, and the encoder block of GNNs as tools for dimensionality reduction. Subsequently, we apply the \gls{DBSCAN} algorithm \cite{ester1996dbscan} to the reduced-dimensional data to identify distinct clusters representing various fluid flow behaviors.
\section{Dimensionality reduction}
Dimensionality reduction is crucial in simplifying high-dimensional data, making it amenable to visualization and data analysis. We focus on PCA and t-SNE, alongside examining how the encoder part of GNNs serves as a useful tool for dimensionality reduction.
\subsection{Principal Component Analysis (PCA)}
PCA is a linear technique that reduces the dimensions of a dataset by transforming them to a new basis where the axes are the directions of maximum variance. It effectively compresses the data while attempting to retain the original variance in the data.

\subsection{t-Distributed Stochastic Neighbor Embedding (t-SNE):}
t-SNE, a non-linear technique, converts the high-dimensional Euclidean distances between points into conditional probabilities that represent similarities, aiming to preserve the local structure of the data. The cost function minimized by t-SNE is based on the Kullback-Leibler divergence \cite{csis} between the distribution of the high-dimensional data and the distribution of the low-dimensional embedding.
\subsection{GNN Encoder}
The encoder segment of a GNN architecture learns a condensed, low-dimensional representation of the graph-based simulation data, facilitating subsequent analysis. Given a graph $G$ with nodes and edges representing complex relationships, the GNN encoder aims to capture the essential information in a lower-dimensional space, referred to as the latent space representation $Z$, which can be expressed as $Z = \text{Encoder}(G)$. This process not only simplifies the data but also retains significant structural and feature-related information crucial for understanding the underlying patterns of the graph.
\section{Clustering with DBSCAN}
Following the dimensionality reduction of simulation data, we can employ the DBSCAN algorithm to identify clusters within the data based on density. DBSCAN is adept at discovering clusters of varying shapes and sizes in a dataset and marking outliers in low-density areas. Its core idea is to classify points as core points, border points, or outliers, based on the density of their neighborhoods, defined by two parameters: $\epsilon$ and minimum points $P$. Given a dataset, DBSCAN forms clusters based on the criterion:
\begin{equation}
\text{If } |N_\epsilon(p)| \geq P
\end{equation}
This indicates that a point $p$ is a core point if at least $P$ other points are within $\epsilon$ distance from $p$, where $N_\epsilon(p)$ denotes the $\epsilon$-neighborhood of $p$.
\section{Experiments and results}
Dimensionality reduction techniques such as PCA and t-SNE were applied to both transitional and steady-state simulation data constituting the training-validation dataset from surrogate modelling, facilitating the visualization and interpretation of complex flow dynamics. PCA and t-SNE are used to transform the target data to 2 dimensions and the results from this transformation can be seen in Figure \ref{pca_sim} and Figure \ref{tsne_sim}. Each data point corresponds to a simulation case and marked by its velocity ratio, which is taken as the ratio of the higher of the two inlet velocities to that of the lower. In the t-SNE plot, we observe clearly distinct clusters or rings of data points on the top left and bottom right corners. The clusters are formed by velocity ratios greater than three, likely indicating the phenomenon of complete adhesion to the Coanda surfaces. Evidenced by the clustering of points with higher velocity ratios, we propose the hypothesis that as the velocity ratio increases, the tendency of the outflow jet to exhibit complete adhesion to one of the surfaces becomes more pronounced. Simulations with lower velocity ratios appear to be more dispersed across the t-SNE plot, which might indicate a more varied flow behavior at these ratios or a weak adhesion effect. The data does not cluster as tightly as those with higher velocity ratios, implying a case of jet deflection without adhesion to surfaces. We also plot the t-SNE data categorizing each data point based on its inlet velocities relationship as seen in Figure \ref{tsne_in}. The application of DBSCAN to these low-dimensional representations from t-SNE transformation provides further evidence to our speculated theories with the identification of distinct flow behavior clusters, based on Coanda adhesion phenomena. 
\begin{figure}[ht]
    \centering
    \includegraphics[width=10cm]{images/Clustering/tsne_sim_cmap.png}
    \caption{Scatter plot visualizing the 2D t-SNE reduction of steady-state simulation data, categorized by varying velocity ratios. Each point on the plot represents a simulation data sample, and the color coding corresponds to the velocity ratios. }
    \label{tsne_sim}
    \end{figure}
\begin{figure}[ht]
    \centering
    \includegraphics[width=10cm]{images/Clustering/pca_sim_cmap.png}
    \caption{Scatter plot visualizing the 2D PCA reduction of steady-state simulation data, categorized by varying velocity ratios. Each point on the plot represents a simulation data sample, and the color coding corresponds to the velocity ratios. }
    \label{pca_sim}
    \end{figure}
\begin{figure}[ht]
    \centering
    \includegraphics[width=10cm]{images/Clustering/tsne_inlet_cmp.png}
    \caption{Scatter plot visualizing the 2D t-SNE reduction of steady-state simulation data, categorized by Inlet 1 and Inlet 2 relationship.}
    \label{tsne_in}
    \end{figure}
An examination of the scatter plot in Figure \ref{dbscan_tsne} shows two clusters as well as outlier points scarcely distributed between these two regions:
\begin{itemize}
  \item \textbf{Yellow Cluster (Coanda Adhesion to the Top Surface):} The yellow cluster, isolated in the lower quadrant of the plot, likely represents instances of Coanda adhesion to the top surface of the nozzle. The spatial separation of these points from the central mass suggests a specific, consistently observed flow behavior across these simulations.
  \item \textbf{Blue Cluster (Coanda Adhesion to the Bottom Surface):} The cluster circled in blue, situated in the upper right of the plot, corresponds to the simulations exhibiting Coanda adhesion to the bottom surface. The compactness and isolated location of this cluster signify a distinct and strong pattern of flow behavior.
  \item \textbf{Purple Outlier Points (Jet Deflection):} The outliers points marked in purple, dispersed between the blue and yellow clusters, likely characterize scenarios where the jet deflects towards either surface, representing a transitional behavior that does not culminate in pronounced Coanda adhesion.
\end{itemize}
\begin{figure}[ht]
    \centering
    \includegraphics[width=10cm]{images/Clustering/dbscan_sim.png}
    \caption{DBSCAN clustering visualized on t-SNE-reduced steady-state simulation data.}
    \label{dbscan_tsne}
\end{figure}
We annotate all the points on the plot to analyze and evaluate our argument. We also perform visual inspection of the simulation cases and classify them into four instances:  
\begin{enumerate}
    \item Complete adhesion to the bottom Coanda surface. 
    \item Jet deflection and partial adhesion to the bottom Coanda surface. 
    \item Complete adhesion to the top Coanda surface. 
    \item Jet deflection and partial adhesion to the top Coanda surface. 
\end{enumerate}
The classification of the cases is presented in the Appendix \ref{appendix}. It is observed that the cluster on the bottom left corresponds to cases displaying complete adhesion to the bottom Coanda surface whereas the one on the top right are constituted by cases that show complete Coanda adhesion to the top wall. We observe a 100\% prediction accuracy in clustering the low-dimensional, steady-state data. Thus, we can assert that simulation cases with higher velocity ratios (above 3 - 3.5) show complete Coanda adhesion to either of the surfaces. Higher Inlet 1 velocity shows adhesion to the top surface and vice-versa. \\
The success of the dimensional reduction and clustering techniques means that one does not have to perform the time-consuming task of manually identifying and classifying the cases where Coanda adhesion occurs. Especially for large datasets, this task becomes laborious and instead can be automated by reducing the dimensionality of the data and performing DBSCAN. \\
% It is to be noted that although PCA is a successful
The PCA-reduced data also shows dense regions formed by cases with higher velocity ratios. However, clustering of only-PCA reduced target data fails to recognize these clusters, demonstrated by Figure \ref{dbscanpca}. This may be attributed to the fact that PCA is a linear dimensionality reduction technique and may sometimes not work effectively for certain datasets when the structure of the data is non-linear. PCA preserves global structure and may not unfold the data in a way that emphasizes the separation between clusters, particularly if the clusters are non-linearly separable in the original high-dimensional space. t-SNE, on the other hand, is specifically designed to preserve local neighborhood structure and can reveal clusters even in complex datasets with non-linear structures. This result led us to perform PCA reduction to 50 components followed by t-SNE reduction to 2 dimensions. This combination was able to replicate similar results to t-SNE-reduced data, as seen in Figure \ref{pcatsnesim}. The outcome of clustering from DBSCAN is show in Figure \ref{dbscanpcatsne}. \\
% The application of DBSCAN to these reduced-dimensional representations led to the identification of distinct flow behavior clusters, categorized based on jet deflection and Coanda adhesion phenomena. While there are not any discernible patterns for the transitional data, we observe clear structures and patterns emerging for steady-state data. The implementation of unsupervised learning techniques, from dimensionality reduction to clustering, unveils the nuanced patterns hidden within CFD simulation data. The dimensionality reduction of simulation data by PCA and t-SNE reduced data undergoes DBSCAN clustering. This presents distinguishable clusters which are indicative of the steady-state data's flow behavior patterns. 

\begin{figure}[ht]
    \centering
    \includegraphics[width=8cm]{images/Clustering/dbscan_pca_sim.png}
    \caption{DBSCAN clustering visualized on PCA-reduced steady-state simulation data.}
    \label{dbscanpca}
    \end{figure}

\begin{figure}[ht]
\centering
\includegraphics[width=10cm]{images/Clustering/pca_tsne_sim_cmap.png}
\caption{Scatter plot visualizing the subsequent PCA and t-SNE reduction of steady-state simulation data, categorized by varying velocity ratios. }
\label{pcatsnesim}
\end{figure}
\begin{figure}[ht]
    \centering
    \includegraphics[width=10cm]{images/Clustering/dbscan_tsne_pca_sim.png}
    \caption{DBSCAN clustering visualized on subsequent PCA and t-SNE reduced steady-state simulation data.}
    \label{dbscanpcatsne}
    \end{figure}
% This plot substantiates the ability of the steady-state data, in contrast to the transitional flow data, to manifest clear and structured patterns. Each point on the plot corresponds to a sample from the training-validation dataset. The annotation of case numbers to each point and the above hypothesis about clusters is validated by visual inspection of cases, i.e; if the Coanda adhesion has occurred and the direction of jet deflection. The resultant categorization of the flow behaviors, based on the clustering outcomes, underscores the unsupervised learning techniques' utility in gleaning meaningful insights from complex datasets. \\

For the transitional state, t-SNE did not reveal any distinct clusters, indicating that at this early stage, the flow behaviors do not exhibit clear patterns or groupings that can be discernibly separated. Similarly, when applying t-SNE and DBSCAN to the latent space representations of the GNN encoder as seen in Figure \ref{fig:dbscan_results2}, the resulting clusters were not as prominent or well-defined as those observed within the steady-state data at 30000 time-steps. The latent space clustering outcome suggests that the encoder captures complex high-level features that may not linearly relate to the flow behaviors characterized at steady-state. This observation emphasizes the challenges inherent in extracting clear patterns from dynamic systems in transition and from abstract representations such as the latent space of a GNN.
\begin{figure}[ht]
    \centering
    \includegraphics[width=10cm]{images/Clustering/cluster2.png}
    \caption{DBSCAN clustering results visualized on t-SNE reduced latent space vectors.}
    \label{fig:dbscan_results2}
    \end{figure}

It is to be noted that t-SNE has an element of randomness in the way it projects high-dimensional data to a lower-dimensional space. This randomness is due to the stochastic nature of the algorithm, particularly in the initial placement of points in the low-dimensional space. The orientation and the exact position of the clusters in the map are not fixed and can differ by a rotation or reflection. This is because t-SNE does not preserve the orientation or the exact distances; instead, it aims to preserve the local structure of the data and the global relationships between clusters, despite the randomness. 