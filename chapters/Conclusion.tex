\chapter{Conclusion}
\label{chap:Conclusion}
This thesis has effectively demonstrated the viability and efficiency of employing Graph Neural Networks (GNNs) as surrogate models for the prediction of stable, steady-state nozzle flow dynamics, marking a significant advancement in the field of Computational Fluid Dynamics (CFD). Through a methodical approach that leverages transient simulation data, this research has offered a groundbreaking alternative to traditional CFD simulations, which are characterized by their computational intensity and extensive time requirements. The findings from this study provide a concrete foundation for the assertion that GNNs can serve as powerful tools in the simulation and analysis of fluid dynamics, specifically in applications related to nozzle flows.
\section{Contributions}
\begin{enumerate}
\item Development of a GNN-Based Surrogate Model: The research successfully developed and optimized a GNN model capable of predicting the velocity and pressure fields in nozzle flow simulations from early, transient states. This surrogate model has proven to effectively use short-term, less computationally demanding simulation results to forecast stable, steady-state flow conditions with remarkable accuracy. This achievement not only demonstrates the technical feasibility of GNNs in fluid dynamics but also positions them as a cornerstone for future CFD simulation methodologies.

\item Comparative Analysis of Model Efficiency: A thorough investigation into the efficiency and practicality of the surrogate model was conducted, comparing its performance against traditional, long-duration CFD simulations. The GNN model exhibited substantial computational savings while maintaining a high level of prediction accuracy. This comparative analysis provided empirical evidence supporting the surrogate model's capability to serve as a viable and efficient alternative to conventional CFD methods.

\item Innovative Application of Clustering Techniques: The thesis explored the use of clustering on low-dimensional data to classify nozzle flow simulations effectively. This approach allowed for an enhanced understanding of the simulation data, leading to more precise predictions of Coanda effect occurrences and facilitating better design and optimization of nozzle configurations. This contribution highlights the potential of integrating advanced data analysis techniques with DL models to improve simulation outcomes.

\item Exploration of Advanced GNN Architectures: The investigation into advanced GNN architectures and training methodologies underscored the potential for further enhancements in model performance. This exploration revealed that incorporating different graph convolutional layers, attention mechanisms, and training strategies could significantly refine the surrogate model's predictive accuracy and efficiency, paving the way for future advancements in the field.

\end{enumerate}


\section{Future directions}

The research presented in this thesis unequivocally asserts the transformative potential of GNNs as surrogate models in the domain of CFD. By achieving significant computational efficiency and maintaining high levels of accuracy in predicting nozzle flow dynamics, this study sets a new benchmark for future research in fluid dynamics simulations. Looking forward, it is imperative to expand upon this work by exploring more complex nozzle configurations, integrating multiscale modeling approaches, and further refining GNN architectures to enhance their generalizability and predictive capabilities across a broader spectrum of fluid dynamics applications. Some possible directions of future research could be, 
\begin{enumerate}
    % Building upon the significant achievements of this thesis, there are several avenues for future research that promise to further advance the application of Graph Neural Networks (GNNs) in Computational Fluid Dynamics (CFD). The integration of GNNs has proven transformative for predicting nozzle flow dynamics, showcasing the potential for more efficient, accurate simulations. To continue this trajectory of innovation, future work should focus on the following key areas:

    \item \textbf{Enhancement of GNN architecture for improved accuracy:} One of the primary objectives moving forward is the development of an improved GNN architecture that enhances prediction accuracy. This involves exploring novel graph convolutional layers, attention mechanisms, and network structures that can more effectively capture the complexities of fluid dynamics. Advanced neural network designs that specifically address the challenges of turbulent flow simulations will be critical in achieving higher levels of accuracy in predicting nozzle flow behavior.
    \item \textbf{Prediction of extended flow quantities of interest:} An additional critical objective for future work is the extension of the GNN model's capabilities to predict a broader array of flow quantities that are crucial for a comprehensive understanding of fluid dynamics. Specifically, the model will be tailored to accurately forecast values such as turbulent viscosity \gls{nut},turbulent kinetic energy k, and the specific rate of dissipation \gls{omega}. These quantities are fundamental in the analysis and modeling of turbulent flows, providing deeper insights into the behavior of fluids in various engineering applications.    
    \item \textbf{Optimization of model training times:} Another crucial area of research is the optimization of the GNN model to reduce training times without compromising accuracy. Techniques such as pruning, quantization, and knowledge distillation could be employed to streamline the model, making it more computationally efficient. Additionally, leveraging parallel computing and advanced hardware accelerators will be explored to further enhance training efficiency, enabling the model to learn from larger datasets in shorter time frames.
    
    \item \textbf{Development of Physics-Informed GNNs:} Incorporating governing equations of fluid dynamics into the model's loss function presents a promising approach to enhance prediction accuracy. By developing a Physics-Informed Graph Neural Network (PI-GNN), the model can leverage both data-driven learning and the inherent physics of fluid flow, ensuring more accurate and physically plausible predictions. This approach not only aids in better capturing the complex phenomena associated with nozzle flows but also in improving the model's interpretability and reliability.
    
    \item \textbf{Transfer learning and model generalization:} Future research will also focus on enhancing the model's generalization capabilities through transfer learning. This involves training the GNN model on a diverse set of nozzle flow scenarios and then fine-tuning it for specific applications, enabling the model to adapt to different nozzle geometries and flow boundary conditions effectively. The goal is to validate the model across a wide range of nozzle configurations and operating conditions, ensuring its applicability and robustness in varied fluid dynamics simulations.
       
\end{enumerate}
In sum, this thesis not only fulfills its objectives but also opens new avenues for research, underscoring the pivotal role of DL in advancing CFD towards more efficient and accurate simulations. The technical achievements documented herein confirm the feasibility of GNNs as powerful surrogate models, signifying a major step forward in our ability to simulate and understand complex fluid dynamics phenomena with unprecedented efficiency and precision.




