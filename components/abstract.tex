% !TEX root = ../main.tex
% The abstract.
% Included by MAIN.TEX
\clearemptydoublepage
\phantomsection
\addcontentsline{toc}{chapter}{Abstract}

\vspace*{2cm}
\begin{center}
{\Large \textbf{Abstract}}
\end{center}
\vspace{1cm}
Turbulent flows, characterized by their complex and chaotic nature, play a pivotal role in various engineering and natural systems. Understanding and analyzing these phenomena is essential for optimizing design, predicting crucial outcomes and addressing real-world challenges. Therefore, obtaining accurate, efficient and rapid predictions of turbulent behaviors is of utmost importance. Data-driven methods such as Machine Learning algorithms are being increasingly implemented to speed up flow predictions compared to numerical solvers. However, these models tend to have poor generalization capabilities and are often restricted to simple geometries on structured grids. Hence, the development and application of a Graph Neural Network (GNN) based framework is proposed to handle irregular unstructured mesh data from turbulent flow simulations. Moreover, a meta-learning approach can be devised that further empowers the GNN-based surrogate model to adapt and generalize more effectively, even when confronted with out-of-distribution data, with the potential to make turbulent flow predictions more accurate and robust. The underlying goal of this research is to leverage the predictions of the surrogate model to perform a comprehensive exploratory analysis of the parameter space that governs the performance of airfoils operating in turbulent flow conditions.
